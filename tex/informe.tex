
\documentclass[a4paper, 10pt, twoside]{article}

\usepackage[top=1in, bottom=1in, left=1in, right=1in]{geometry}
\usepackage[utf8]{inputenc}
\usepackage[spanish, es-ucroman, es-noquoting]{babel}
\usepackage{setspace}
\usepackage{fancyhdr}
\usepackage{lastpage}
\usepackage{amsmath}
\usepackage{amsfonts}
\usepackage{amsthm}
\usepackage{verbatim}
\usepackage{fancyvrb}
\usepackage{graphicx}
\usepackage{float}
\usepackage{enumitem} % Provee macro \setlist
\usepackage{tabularx}
\usepackage{multirow}
\usepackage{hyperref}
\usepackage{xspace}
\usepackage{tikz} % Provee herramientas de dibujado para subrayado
\usepackage{ulem} % Provee subrayados punteados \dotuline
\usepackage[toc, page]{appendix}


%%%%%%%%%% Constantes - Inicio %%%%%%%%%%
\newcommand{\titulo}{Trabajo Práctico 1}
\newcommand{\materia}{Bases de datos}
\newcommand{\integrantes}{Delgado · Lovisolo · Petaccio · Rebecchi}
\newcommand{\cuatrimestre}{Segundo Cuatrimestre de 2014}
%%%%%%%%%% Constantes - Fin %%%%%%%%%%


%%%%%%%%%% Configuración de Fancyhdr - Inicio %%%%%%%%%%
\pagestyle{fancy}
\thispagestyle{fancy}
\lhead{\titulo · \materia}
\rhead{\integrantes}
\renewcommand{\footrulewidth}{0.4pt}
\cfoot{\thepage /\pageref{LastPage}}

\fancypagestyle{caratula} {
   \fancyhf{}
   \cfoot{\thepage /\pageref{LastPage}}
   \renewcommand{\headrulewidth}{0pt}
   \renewcommand{\footrulewidth}{0pt}
}
%%%%%%%%%% Configuración de Fancyhdr - Fin %%%%%%%%%%


%%%%%%%%%% Miscelánea - Inicio %%%%%%%%%%
% Evita que el documento se estire verticalmente para ocupar el espacio vacío
% en cada página.
\raggedbottom

% Separación entre párrafos.
\setlength{\parskip}{0.5em}

% Separación entre elementos de listas.
\setlist{itemsep=0.5em}

% Asigna la traducción de la palabra 'Appendices'.
\renewcommand{\appendixtocname}{Apéndices}
\renewcommand{\appendixpagename}{Apéndices}
%%%%%%%%%% Miscelánea - Fin %%%%%%%%%%

% Macro para subrayar PK y FK
\newcommand{\pkfk}[1]{
    \tikz[baseline=(todotted.base)]{
        \node[inner sep=0pt,outer sep=1.5pt] (todotted) {#1};
        \draw (todotted.south west) -- (todotted.south east);
        \node[inner sep=0pt,outer sep=3pt] (todotted) {#1};
        \draw (todotted.south west) -- (todotted.south east);
    }
}

\newcommand{\fk}[1]{
    \tikz[baseline=(todotted.base)]{
        \node[inner sep=1.5pt,outer sep=0pt] (todotted) {#1};
        \draw [dashed] (todotted.south west) -- (todotted.south east);
    }
}

\begin{document}


%%%%%%%%%%%%%%%%%%%%%%%%%%%%%%%%%%%%%%%%%%%%%%%%%%%%%%%%%%%%%%%%%%%%%%%%%%%%%%%
%% Carátula                                                                  %%
%%%%%%%%%%%%%%%%%%%%%%%%%%%%%%%%%%%%%%%%%%%%%%%%%%%%%%%%%%%%%%%%%%%%%%%%%%%%%%%


\thispagestyle{caratula}

\begin{center}

\includegraphics[height=2cm]{DC.png} 
\hfill
\includegraphics[height=2cm]{UBA.jpg} 

\vspace{2cm}

Departamento de Computación,\\
Facultad de Ciencias Exactas y Naturales,\\
Universidad de Buenos Aires

\vspace{4cm}

\begin{Huge}
\titulo
\end{Huge}

\vspace{0.5cm}

\begin{Large}
\materia
\end{Large}

\vspace{1cm}

\cuatrimestre

\vspace{4cm}

\begin{tabular}{|c|c|c|}
\hline
Apellido y Nombre & LU & E-mail\\
\hline
Delgado, Alejandro N.  & 601/11 & nahueldelgado@gmail.com\\
Lovisolo, Leandro      & 645/11 & leandro@leandro.me\\
Petaccio, Lautaro José & 443/11 & lausuper@gmail.com\\
Rebecchi, Alejandro & 15/10 & alejandrorebecchi@gmail.com\\
\hline
\end{tabular}

\end{center}

\newpage


%%%%%%%%%%%%%%%%%%%%%%%%%%%%%%%%%%%%%%%%%%%%%%%%%%%%%%%%%%%%%%%%%%%%%%%%%%%%%%%
%% Introducción                                                              %%
%%%%%%%%%%%%%%%%%%%%%%%%%%%%%%%%%%%%%%%%%%%%%%%%%%%%%%%%%%%%%%%%%%%%%%%%%%%%%%%

\section{Introducción}
Presentaremos una solución para el problema de verificación y conservación de los resultados de las elecciones
de los diferentes cargos políticos que presenta una universidad, tomando como guía, el estatuto de la Universidad
de Buenos Aires \footnote{www.uba.ar/download/institucional/uba/9-32.pdf}.

El problema en cuestión contempla una serie de restricciones sobre como se realizan las votaciones y sobre quienes
pueden presentarse para cada cargo que formarán parte de la verificación a modelar. Una vez resueltas las restricciones,
se podrá validar los datos que se introduzcan, obteniendo registro de tanto los diferentes actores en las elecciones, como
los cargos que tendrán los ganadores y para estos, los votos que recibieron, guardando quienes los realizaron solo para la elección
que corresponda.

Utilizaremos las herramientas vistas en la materia, el modelado basado en el Diagrama de Entidad Relación, su
MR resultante y la base de datos final que presentaremos en SQLite.


\newpage


%%%%%%%%%%%%%%%%%%%%%%%%%%%%%%%%%%%%%%%%%%%%%%%%%%%%%%%%%%%%%%%%%%%%%%%%%%%%%%%
%% DER                                                                       %%
%%%%%%%%%%%%%%%%%%%%%%%%%%%%%%%%%%%%%%%%%%%%%%%%%%%%%%%%%%%%%%%%%%%%%%%%%%%%%%%


\section{Diagrama de Entidad Relación}


%%%%%%%%%%%%%%%%%%%%%%%%%%%%%%%%%%%%%%%%%%%%%%%%%%%%%%%%%%%%%%%%%%%%%%%%%%%%%%%
%% MR                                                                        %%
%%%%%%%%%%%%%%%%%%%%%%%%%%%%%%%%%%%%%%%%%%%%%%%%%%%%%%%%%%%%%%%%%%%%%%%%%%%%%%%


\section{Modelo Relacional}

\textbf{Facultad}(\underline{idFacultad}, nombre) \\
PK = CK = \{idFacultad\}

\textbf{Empadronado}(\underline{DNI}, nombre, fechaDeNacimiento, tipo) \\
PK = CK = \{DNI\}

\textbf{Estudiante}(\pkfk{DNI}, fechaDeInscripcion) \\
PK = CK = FK = \{DNI\}

\textbf{Graduado}(\pkfk{DNI}, tipo) \\
PK = CK = FK = \{DNI\}

\textbf{Graduado de la UBA}(\pkfk{DNI}) \\
PK = CK = FK = \{DNI\}

\textbf{Graduado de otra universidad}(\pkfk{DNI}, inicioActividades) \\
PK = CK = FK = \{DNI\}

\textbf{Profesor}(\pkfk{DNI}, nacionalidadUniversidad, tipo) \\
PK = CK = FK = \{DNI\}

\textbf{Profesor regular}(\pkfk{DNI}, \fk{nacionalidadUniversidad}) \\
PK = CK = \{DNI\}
FK = \{nacionalidadUniversidad\}

\textbf{Consejero Directivo}(\pkfk{DNI},\underline{períodoConsejeroDirectivo}, \fk{idAgrupación}, tipo) \\
PK = CK = \{(DNI, períodoConsejeroDirectivo)\}
FK = \{DNI, idAgrupación\}

\textbf{Consejero Directivo por el Claustro de Profesores}(\pkfk{DNI}, \pkfk{períodoConsejeroDirectivo}, \fk{idAgrupación}) \\
PK = CK = \{(DNI, períodoConsejeroDirectivo)\}
FK = \{(DNI, períodoConsejeroDirectivo), idAgrupación\}

\textbf{Consejero Directivo por el Claustro de Graduados}(\pkfk{DNI},\pkfk{períodoConsejeroDirectivo}, \fk{idAgrupación}) \\
PK = CK = \{(DNI, períodoConsejeroDirectivo)\}
FK = \{(DNI, períodoConsejeroDirectivo), idAgrupación\}
	
\textbf{Consejero Directivo por el Claustro de Estudiantes}(\pkfk{DNI},\pkfk{períodoConsejeroDirectivo}, \fk{idAgrupación}) \\
PK = CK = \{(DNI, períodoConsejeroDirectivo)\}
FK = \{(DNI, períodoConsejeroDirectivo), idAgrupación\}

\textbf{Agrupación política}(\underline{idAgrupación}, nombre) \\
PK = CK = \{idAgrupación\}

\textbf{sePresentaDurante}(\pkfk{idAgrupación},\pkfk{fecha}) \\
PK = CK = \{(idAgrupación, fecha)\}
FK = \{idAgrupación, fecha\}

\textbf{Calendario Electoral}(\underline{fecha})

\textbf{Decano}(\pkfk{DNI}, \underline{períodoDecano})

\textbf{VotoADeacno}(\pkfk{DNIDecano}, \pkfk{períodoDecano}, \pkfk{DNIConsejeroDirectivo} \pkfk{períodoConsejeroDirectivo})

\textbf{Consejero Superior}(\pkfk{DNI}, \underline{períodoConsejeroSuperior}, tipo)

\textbf{VotoAConsejeroSuperior}(\pkfk{DNIConsejeroSuperior}, \pkfk{períodoConsejeroSuperior}, \pkfk{DNIConsejeroDirectivo} \pkfk{períodoConsejeroDirectivo})

\textbf{Consejero Superior por el Claustro de Profesores}(\pkfk{DNI},\pkfk{períodoConsejeroSuperior})

\textbf{Consejero Superior por el Claustro de Graduados}(\pkfk{DNI},\pkfk{períodoConsejeroSuperior})

\textbf{Consejero Superior por el Claustro de Estudiantes}(\pkfk{DNI},\pkfk{períodoConsejeroSuperior})

\textbf{Rector}(\pkfk{DNI}, \underline{períodoRector})

\textbf{fueVotadoPorConsejerosDirectivos}(\pkfk{DNIRector}, \pkfk{períodoRector}, \pkfk{períodoConsejeroDirectivo}, \pkfk{DNIConsejeroDirectivo})

\textbf{fueVotadoPorDecanos}(\pkfk{DNIRector}, \pkfk{períodoDecano}, \pkfk{períodoDecano}, \pkfk{DNIDecano})

\textbf{fueVotadoPorConsejerosSuperiores}(\pkfk{DNIRector}, \pkfk{períodoRector}, \pkfk{períodoConsejeroSuperior}, \pkfk{DNIConsejeroSuperior})

\end{document}
